\documentclass[a4paper,10pt]{IEEEtran}
\usepackage[latin1]{inputenc}
\usepackage[pdftex,colorlinks,linkcolor=black,citecolor=black]{hyperref}
\usepackage{amsmath}

\author{\IEEEauthorblockN{R\"udiger Ehlers \\}
\IEEEauthorblockA{University of Bremen \& DFKI GmbH\\
Germany
}}

\begin{document}
\title{Approximately Propagation Complete and Approximately Conflict Propagating SAT Encoding Computation MaxSAT Benchmarks}
\maketitle

\section{Description}
This benchmark set contains MaxSAT instances that encode the problem of finding approximately propagation complete and approximately conflict propagating SAT encodings for a couple of interesting constraints. The approach for reducing this problem to MaxSAT is explained in a paper \cite{EP2018} to be published at the 21$^\mathit{st}$ International Conference on Theory and Applications of Satisfiability Testing (SAT 2018).

The benchmarks set is based on encoding the constraints used in the experimental evaluation of the paper mentioned above , where constraint encodings for the progagation quality tuples $(\infty,\infty)$, $(1,\infty)$, $(2,\infty)$, $(3,\infty)$, $(3,3)$, and $(\infty,1)$ are computed. The meaning of these tuples is also mentioned in this paper. 

To keep the benchmark set small, a couple of benchmarks for which both the solvers \texttt{LMHS} \cite{DBLP:conf/sat/SaikkoBJ16} (in the version from March 2018) and \texttt{maxino-2015-k16} \cite{DBLP:conf/ijcai/AlvianoDR15} both need less than 3 seconds of solving time on a moderately modern computer (using an Intel(R) Core(TM) i5-4200U CPU with 1.60\,GHz computation speed) have been removed. Furthermore, MaxSAT instances whose computation takes more than 30 minutes on the said computer are also left out.

The benchmark file names contain:
\begin{itemize}
\item the name of the constraint that is to be encoded into conjunctive normal form, and
\end{itemize}
\newpage
\begin{itemize}
\item the propagation quality tuple, where the elements are concatenated and $\infty$ elements are replaced by ``\texttt{99}''.
\end{itemize}
Some of these constraints are taken from the set of examples supplied with the GenPCE tool by Brain et al.~\cite{DBLP:conf/vmcai/BrainHKM16} for computing propagation SAT complete encodings.

The program to compute the MaxSAT benchmarks is available at \url{https://github.com/progirep/optic} in the branch ``\texttt{MaxSATEvaluationBenchmarkGeneration}''.

\bibliographystyle{IEEEtran}
\bibliography{description}

\end{document}